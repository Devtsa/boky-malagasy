no tapahin' ibaba. Mahatazana ny lavitra aho, mahatsinjo ny akaiky. Ny
biby anaty rano aza mikafay ahy, ka tsy misy hitaninany andro; ary ny
aumbony vohitra tsy misy hitsindrohana, fa matahotra ahy, ka inona akory
izay Ibonia kely vao teraka iny.»

Fa nony nahita an' Ibonia nihetsika tao anaty afo Ranakombe, dia
nitso-drano azy hoe : «Akory iha, ry Ibonia ! Io lahy no Iboniamasibonia-
manoro : manoro ny lanitra, mandrehitra ny tany; manoro ny lavitra, fa tsy
manoro ny akaiky, manoro ahy. Mitaky mahazo, itakian-tsy ahazoana;
toka-mahamamo tany. Izay manarangarana azy any an-danitra sy izay
mifosa azy, amin' ny lafin-tany efatra anampiny azy koa. Ataon' ireo fa
hahasakana azy izy, kanjo ataony indray mifaoka avokoa izy rehetra, io no
laby sarotra amin' ny tany >

Dia nihiaka Ibonia ka nanao hoe : «Eja nanjary ahy lb Dia nitsipika
niala tao anatin' ny afo indray izy, ka vakívaky ny tany nalehan' ny ton-
gony, tomandeboka ny tany, maina am-potony ny hazo, ritra avy ny rano;
ka dia molangena (10) avokoa ny vahoaka niangona te0 amin Tbonia. Dia
nosafoiny reniny sy ny vahoaka izay maty tamin' ny nahaterahany ka velona
soa aman-tsara avokoa izy rehetra.

Raivato kosa tamnin' izay dia intelo niantombina tany Mananivo. Fam-
bara loza no niseho, fa niherikerika ny lanitra teny amin' Andriamanitra,
nangorohoro ny tany, moramorana ny Asara, (11) nandrivodrivotra Itsiaiso-
tra; (12) nanihikihy (13) ny lohataona, fa nolalovan-dahy sarotra; nanasa-
rotra ny an-tena, nanamora ny an' olona; nefa ny azy dia azy rahateo.

Ary izao no baiko nataon-dRanakombe tamin' Ibonia, rehefa hody izy:
«Hanan-tombo anie ny herin' io zaza io, raha ampiako hoe Isikindahy koa
ny anarany ! Meda io zaza io, fa hataoko Isikindahindraifosalabibehatoka
no anarany. Iraisimbeforerarehefahentikanivaniany, Imihabenitratrany,
miray voavato ny iafaran' ny lalaony, misikin-dahy ila rano (14) ny tenate-
nany. Izany no sikin-dahy sarotra>. <Tsy tiako izany, ry baba», hoy Ibonia,
cfa sikin' ny mpangalatra; fainga-miamboho ny sikin-dahim-pandositra; tsy
mba manatrika, fa mitoe-dahy mavozo; koa izay tojo azy no irangotany,
na vato na hazo; ary mamonjy zohy faingana izy, fa izany no ataony
fandosirana, ry baba>.

Dia novan-dRanakombe indray ny anarany ka hoy izy : <Hataoko hoe
Isikindahinandriampapangolahilavelatra ny anaran' ialahy: mikenakena
miadana, mifaofao niambolom-body. Izany no sikin-dahy tsy mena-badin
olona; ny tongony kosa no mandray, fa tsy hba ny tanany. Izany no
mpiebo amin 'ny tany, «Aky I tsy tiako izany, ry baba», hoy Ibonia, cfa
sikin-dahy lany elatra, ka ny tongotra manitsaka ny ratsy no enti-mandrairay
hanina>.

«Hataoko hoe Isikindahínandriantolohoboboka àry no anaran' ialahy:
mísidin-tsy avo, mandeha tsy be, mitsororo-pihinana, manta fanao baran-
dro; (15) iketraham-beny arivo, filo volany (16) zato; nakopakopany ilay

_________________________________________________________________________________________________
(10) Lampana, noana. 
(11) volana Fandroana. 
(12) Ririaiaa. 
(13) Migaingaina. 
(14) Reraka
(15) Toala. 
(16) anjal-balafotsy.
