taolany sy momoka ny nofony; ka raha voan'ireny andavan-taona, dia
maty andavan'andro; ary raha velomin'ireny andavan-taona, dia velona
andavan'andro. Fa ombilahy sakoko (18) tandroka ireny ka sarotra iadiana:
raha an-doha-tandrony voateviny; am-balanoranony voatsakony; an-tsofiny
voakopany; amin' ny rambony voahifiny; ambany kitrony voahitsany
ary raha an-orony mibaibay. (19) Ireny no lahy sarotra (20) amin' ny tany.
"Tsia, Adarà", hoy ny zaza, "fa zaza ireny, ka tsy mahasakan-dahy izay
hovonoiko".

Dia hoy indray reniny: "Angaha Ingarabelahy sy Izatovotsiota ho sahy hanao izany,
fa ireny no taranak'Imahavory, zanak Ibefihary, lahy sarotra amin'ny tany,
zanak' Ivonointsimatimamonomahafaty. Toradefon'Ingarabelahy: raha voan'ny lelany,
tsy vanona; raha voan'ny zarany, manindao tratra; ary ny alehan'ny antsetra mandripaka arivo".
Fa hoy ny zaza : "Tsia, Adarà, fa tsy mahavadika ny loha tafika entiko izany,
fa zaza. Antsoy ny tampon'Ileolava, fa lasan-dahy sarotra masin'ody
Iampelasoamananoro, fa fasan'Ivatolahiloza. Antsoy hoe : "Lasan' Ivatolahiloza, lasan' Ivatolahisarotra !"
Koa izy loza, izaho antambo", Dia nivoaka namonjy hamerina an'Iampelasoamananoro Ileolava, fa tsy nahazo
azy. Dia hoy ny zaza : "Tsy azo lampelasoamananoro, Adarà, fa indreto miverina Ileolava",



TOKO IV. -ZAZA NANOTO

Nony afaka folo taona, sendra nanatrika ny mpamboly Rasoabemanana,
dia nihetsi-jaza teny izy. Ary hoy nv zaza: "Ento mody aho, Adarà, fa tsy
ravin-tàko aho fa hitàko; tsy ravin'ovy aho ka hanovy; tsy lambo aho
ka hibohibohy; tsy trandraka aho ka hiborobodaka; tsy alika aho ka ho
sarotra aman-delany; tsy aketa aho ka hietaketaka; tsy sokina aho ka ho
faty amam-bolony; tsy vato aho ka hikodiadia; tsy ontsy aho ka tsiomeroa;
tsy zavona aho ka hanenika ny tany; tsy akoholaby aho ka hahatsiaro andro
maraina; tsy akanga aho ka hifaoha anaka; tsy voay aho ka hiandry fitàna.
Fa izaho no saonjo an-kihin-karana: (1) tsy hadim-body, tsy hanin-dravina;
raha dikai-mahabe lohalika, harihi-mahagoa-maso, tondroi-mahamondry
tanana, atono antambo, andrahoin-doza. Ary koa, vahivoraka an-dafy aho:
lolohavi-mahasola, dikai-mahabe kibo, avela mahamondry tongotra. Voay
be miandry fitàna aho: setrain-daka-mandrendrika, robohi-manehi-baniana.
Trano be tazana aho, ka na maro aza tsy ahazoana an-entana, fa
raha ahazoana mandratra; koa raha mitazana ahy ny any an-dafin'ny
riaka, dia anampiko (2) ahy koa".

Dia nentiny tery amin' ny sola-bato Rasoabemanana hizaha izay hiterahany. 
Ary hoy ny zaza tao an-kibony: Tsy tiako ity, Adarà, fa fanatodizam-papango sy fiteraham-boromahery. 
Fa ny papango homana akoho-kely, sady mirehareha homana eny ambonin'ny tompony; ary na dia ompa
aza no manaraka azy, vao mainka mahabe aina azy hany. Ny voromahery
hitako fa mahery, ka izay faohiny, - ny tsy voa dia mifaifay, (3) ary ny voa
dia tsy vanon-ko raha; (4) ny mby an-tanany tsy mba mety latsaka. Koa
dia ento miainga aho, Adarà, fa halako ity tany ity, satria tanin'ny mpihako : (5) mahery fa miasa tanana; 
tsy manam-bahoaka hinaina, tsy manam-bahoaka hirahina; koa dia tsy misy hiandrianana izany", 
Dia niala teny reniny, ary notsipahiny ny vato ka folo ambakambaka (7).

_________________________________________________________________________________________________
(18) Jolko. 
(19) Mahamay, idiran-doza. 
(20) -Loza, mahery.
Toko IV (1) An-tsefatsefa-bato, 
(2).Resiko ka tonga namako 
(3) Fay. 
(4) Zavatra.
(5) Mpiangolangola, mpangalatra. 
(6) Ho ehina. 
(7) Vakivaky.