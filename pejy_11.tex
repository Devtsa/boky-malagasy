Ary nongoahany indray reniny ho eny an-tampon' ny tendrombohitra
avo dia avo, ka hoy izy : <fiako ity, Adará, ka eto aho no hateraka; fa avo
noho ny be ity, ka andrandrain' ny maro, ary misy betsaka hantsoina». Nefa
nivadika indray ny zaza ka nanao hoe : «Ny soa anie ity dia soa, Adarà,
nefa sady tanin-dolo no tanin' angatra, ka izany no tsy itiavako azy; fa ny
tsara mitsoka eto, ny ratsy mitsoka eto». Koa dịa niainga indray izy sady no-
tsipahiny iny tendrombohitra iny, ka tomandeboka (8) ny tany, ary nikodia-
dia sady torotoro ny vato.

Rehefa izany dia niainga indray izy, ka nentiny nitety izao ravin-kazo
rehetra izao reniny, sady niteny izy hoe : «Tsy tiako ity, Adarà, fa arivo
lahy no manjaka : ny be tarihin' ny kely; ka tsy misy fiasana izao, fa
tanin-tsifaka, (9) tanim-barika, tanin-damb0, tanim-biby; ka ento mody
aho, Adarà, fa halako ity, satria tanin' ny mpikolialia>. Koa dia niainga
indray izy, sady notsipahiny ny hazo, ka folo montsamontsana sy niombo-
tra amam-potony ary nianga amam-pakany.

Dia lasa indray izy mianaka nilentika tao anaty rano hitady izay haha-
terahany, ka hoy indray ny zaza : «Tsy tiako ity, Adarà, fa maro biby
malama, ka tsy hita horaharahaina, tsy hisy hiandrianako>. Dia nievotra
indray izy ka nitsoaka tao anaty rano, sady nokobahiny ny rano ka ritra; ary
ny biby tao nokobakobahin' ny riaka, ka nitsinkafona teny ambonin ny
rano. Dia hoy izy : «Tiako ity, Adarà fa fandria-malemy lafika, ka tsy
itadiavako lafika sy ondana. Nefa tsy tiako indray ity, fa tsihy be tsy azo
ahorona, lamba soa tsy azo asafotra, ondam-be tsy enti-mandeha». Raha
vao tapitra izany teniny izany, dia niala indray izy, ka notehafiny ny rano
vao lasa nody izy; koa izany, hono, no nampanonjanonja ny riaka (10).
Ary any an-tanàna indray reniny no namantana.

Nony efa tafakatra reniny dia nasainy nankeny am-bovonana, ka hoy
izy: «Tiako ity, Adarà, fa arivo lahy miroza (11) sy telo lahy mitokana. (12)
Ity no maro manaja sy maro manohana. Nefa tsy tiako indray ity, fa maro
manohana loatra, ka tsy mahatokam-panjakana ahy, satria andevolahy ny
ankamaroany eto, ka mahery vava be ihany, hany ka tsy misy hilatsahan' ny
rariny, ka ento midina aho, Adarà>,

Dia namantana teny amin' ny tapenaka atsimo indray reniny, nefa
mbola nolaviny ihany hoe : «Halako ity, Adarà, fa famonoana omby raha
misy maty, sy fanariam-paditr izay velona; ka ento aho ho any amin' ny
tapenaka avaratras.

Nony tonga teny amin' ny tapenaka avaratra indray reniny, dia hoy
Izy : «Halako ity, Adarà, fa fiantsoana an-Janahary : soa anie izy dia soa;
fa am-pita dia mpianadahy: ny tany no impito fitaka (13) izao, fa tsy miady
an-kitsy loha. Ento eny amin' ny lavan-trano atsinanana aho»,

Rehefa tonga teny reniny, dia izao indray no nambarany : Na soa
aza anie ity, Adarà, dia fisaoran' ny atsimo, fisaoran' ny andrefana, fa
mandry lava maro manainga; (14) koa tsy hitako na rariny izao andriany
lava izao na inona>.

«Ento aho ho any amin' ny sakamandimbiny, fa io no soa anivon' ny
alin-dahy manelo, arivo lahy milanja, zato lahy mihaga. Izaho dia zazalahy
sarotra, koa handrava tanin-drafin' olona mbamin' ny ahy dia ahy rahateo.

_________________________________________________________________________________________________
(8) Lesoka lavalava 
(9) Karazan' amboanala. 
(10) Ranomasina. 
(11) Mikiraviravy. 
(12) Ny andry telo sy ny loba-trano. 
(13) Misy Ataka mnatetika. 
(14) Amin' ny trano kotona ny lavany mítingina eo ambonin' ny zana-kazo no atao hoe mandry lava maro manainga.