tiany indrindra nataony hoe Itiahita; ny faharoa nataony hoe Itsimiasaro.
baka; ny fahatelo Imahalanantsaha, ary ny fahefatra Itiaranovola.

Indray andro raha niara-nipetraka nifanorona tamin' i Endriavelo raha.
vavin-dreniny teo am-bavahady Ibonia, dia hoy izy : <Ry Endriaveļo 1
mitehin-ko lohataona ity ny andro, ka miongy (1) ny vahoaka, fa izaho kosa
izany manina an' lampelasoamananoro vadiko>.

Fa nokiviny kosa ka nataony hoe: «Aoka itsy, anaka, fa hianao miha-
tsaika, (2) mifelatanam-behivavy; Raivato anie ka soriba, Raivato anie
ka varatra è l> Esy l> hoy Ibonia, canao anie aho ka mora, fa raha an'
olona anie aho ka sarotraè l Izaho trano be : tazana, tsy azahoana, fa raha
azahoana, mandratra>.

Fa nanambosy azy kosa ireo andevony ka nanao hoe: *Milomanosa
hianao, aba, milomanosa hianao; hianao no teraky ny volana, zafin' ny
masoandro erỳ anivon' ny lanitra; teraky ny tsy endrikendrehina ho andria-
na, fa andrian-drahateo, zafin' Itsilanitantara; koa masina hanjelajelatra (3)
eny Ileolava>.

«Aky lahim-be !> hoy Ibonia, <miandra aho, vaky ny lanitra; mion-
drika aho, forehitra ny tany; mibirioka. aho, may ny ala atsinanana; hitsa-
hiko ity tany Ileolava ity, miporopotaka(4)milohan-drere». Dia dongodonin'
ny tongony ny tany ka niezinezina hatrany Mananivo izay nitoeran-Raivato
rafilahiny; hany ka nianjera terỳ ny entany teny an-toeran-entany. Raiki-
tahotra amboný ihany Raivato ka niteny hoe :Ey I tsy hitako lahy izay
hiafaran' ity tany Mảnanivo ity, fa ho resin- dahy iraika sady masin' ody
no ozatina angaha izy ity. Koa ny manam-bondraka dia mamonoa, ny
mana-mahia manakaloza l>

Fa notebahin' Andriambahoabesofina rainy izy ka nataony hoe : Ny
haosan 'ialahy I fa haninona no ho resin-dahy iraika sady masin' ody no
ozatina eo ny tany ls «A l> hoy ny zanany, satao ho mpahalala ibaba, ta
tsy mahalala; atao ho mpitadidy, fa tsy mahatadidy>. Dia nangina tsy
niteny intsony izy, fa lasa niolonolona irery, satria notamin-jo.

Nony tamin' ny indray andro dia nifanorona teo an-trano indray
Ibonia sy Endriavelo, ka niteny Ibonia hoe : «Nofofoiko fahatsaika (5)
Jampelasoamananoro, novinadiko fahazaza; koa maty tsy airitro aman-tany,
velona tsy omeko lahy». Fa hoy no navalin' i Endriavelo : Aoka re, anaka,
fa tsy dia hany vehivavy halainao ho vady iny tsy akory».

Fa hoy kosa indray ny fanambosin' ireo andevony : «Milomanosa
hianao, aba, milomanosa hianao, aba; koa masina hianao, ry Ibonia matahl
manana, (6) matahi-manarivo, tsiota (7) tsy adika, fa saonjolahy nateln
Ileolava ka hitaly (8) any Mananivo».

«Aky lahim-bel> hoy indray ny thiak' Ibonia, <ny anaran' ity sabo-
hako ity, izay ho entiko miady amin-dRaivato, dia ataoko hoe Isabohama-
romitoky, lefomaromahay, Itefialahady, Isomorinalatsinainy, Iatreboan
tanetiamantrandraka, Iatreboamandranoaminamalona. Saonjolahy an-kihi-
karana ity: tsy hadim-body, tsy alain-dravina; koa raha tsy rivotra no
manolaola (9) azy tsy miavotsa (10) izy. Ny anaran 'ity andronako (11) ity

_________________________________________________________________________________________________
(1) Miasa, mihady. 
(2) Mihazava, vavivavy. 
(3) Mamirapiratra. 
(4) Manjary fotaks. * io Raivato io ilay naka an-keriny an' Iampelasoamananoro ho vadiny, dia ilay atao hoe Ivato-lahiloza, 
(5) Fony vao torontoronina. 
(6) Manam-be dia be. 
(7) Enina. 
(8) Mandra ndrana
(9) Mamotofoto. 
(10) Miongotra. 
(11) Famaky.
