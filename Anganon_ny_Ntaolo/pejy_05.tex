
I.ANGANO SY ARIRA

                                         IBONIAMASIBONIAMANORO

TOKO I. - FANANAM-BE TSY NAHAFA-PO

Nihotrakotraka, (1) hono, Andriambahoaka atsinanana, fa hamangy an-dRailanitra raibeny, ka nitondra ny zanany, 
dia Ingarabelahy sy Izato votsiota (2) sy Andriambavitoalahy ary Imaroahina sy Imaromanaiky vahoakany. 
Koa dia nitondra ombilahy folo sy reni-omby folo ampianahany nalainy tamin' Itanimaroanio sy Isakatriniba izy.

Ary dia nihotrakotraka toy izany koa Andriambahoaka avaratra, ka nitondra an' Imbolahongeza sy Irainingeza.

Nihotrakotraka toy izany koa Andriambahoaka andrefana, ka nitondra an'Impandrafitrandriamanibola 
sy Imbolahongeza ary ny zanany valo vavy.

Ary mba nanao toy izany koa Andriambahoaka atsimo ka nitondra an' Ikabikabilahy sy Ifosalahibehatoka 
ary ny zanany valo vavy.

Io Railanitra io dia niteraka an' Ifanarangarandanitra; ary izy kosa niteraka azy dimy lahy, 
dia Andriambahoaka atsinanana, Andriambahoaka andrefana sy Andriambahoaka avaratra sy Andriambahoaka atsimo 
ary Andriambahoaka afovoan' ny tany.

Raibeny hovangiany kosa dia nikasa hamahana azy, ka dia nangalany ny ombiny natao hoe Ibetsiviliana 
sy Itahontàka. Ary toy izao no fiteniny nony efa tafahaona tamin' ireo zafiny ireo : 
«Faly aho fa tsy holovàn-dambo amin'alika, lehiretsy, ny taniko ». 
Dia nanao lanonana lehibe izy, ka nanapoaka tafondro sy basy ary nanao fahana ho azy ireo.

Dia avy kosa Andriambahoaka (Ibemampanjaka), afovoan' ny tany sy Rasoabemanana vadiny;
ka dia nihotrakotraka ny lanitra, ary nihorohoro
ny tany izay nolalovan' izy sy ny vahoakany, dia larivolahimihina sy Izato-lahimihaga; 
nizaozao ny akata, (3) nihaon-tendro ny tsipolitra, ary toman-deboka (4) ny tany nolalovan' Itsiazonantso. 
'Ndeha hamangy an-dRailanitra koa izy, fa izy no zokiny indrindra tamin' ireo.

Toy izao kosa no fitenin-dRailanitra, raha tafahaona tamin' Andriambahoaka afovoan' ny tany : 
«Faly aho, leitsy, vangianareo mivady ; kanefa izao no ataoko aminareo: ny basy sy ny tafondro izay 
efa napoaka tamin'izy efa-dahy teo aloha dia veromay no fanapoakako azy: ary ny fahana izay
natolotro ho azy dia ny zavatra izay tsy bada no nomeko azy, satria maro anaka izy. 
Fa inty kosa ny aminareo mivady: ny basy sy ny tafondro izay
hapoakako dia vato (5) tokana no ataoko aminy, nefa atohoko amin' ny tany no fanapoakako azy ».
________________________________________________________________________________________________
(1) Somaritaka. 
(2) Enina. 
(3) Maina ny ahitra. 
(4) Nihalesoka lavabe 
(5) Bala.
