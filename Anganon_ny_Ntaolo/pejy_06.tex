Ary naniraka izy nampaka ny omby hataony fahana, ka hoy izy : 
«Alao lahy Igogoka, (6) alao Itolohobefeo, alao Izatotsiandrohy, alao Izatotsiandravina, alao Iraikopaka, 
alao Irailily, alao Itsiarindahimavozo, alao Iraijao, alao langinimanga, alao Ielakelakinilavitra, 
alao Itsitarani-
marina, alao Ihiontsinamalona, alao Itsikaranoampombo, alao Iketriketrikantendany, alao Imanarivoampeo, 
alao Itranompodiantandroka,
alao Imandraifakoandoha. 
Ary angalao ombilahy zato sy omby saikatra (7) zato koa, sy ondrilahy zato sy ondry saikatra zato, 
sy vorona saikatra zato ary akoho saikatra zato ».

Dia naniraka indray Railanitra hoe: "Alao lahy Ifanarangarandanitra, fa tsy teo izy, 
hanolorako y fahan' ny zaza, dia Andriambahoaká afovoan'ny tany sy Rasoabemanana vadiny". 
Nony tonga izy, dia nantsoiny hanatona azy ireo zanany dimy lahy ireo.
Rehefa izany dia nivoaka Railanitra ka nipetraka tamin' ny fiara volamena, dia nikabary hoe : 
"Efa vitako ny fahan' izy efa-dahy, fa ny an' Andriambahoaka afovoan' ny tany no tsy mbola efako, 
ka aoka hifana-trehantsika, dia ombilahy zato sy omby saikatra zato, ets. Ary ny basy sy ny tafondro napoakako 
dia nasiako vato tokana, ka levona foana tamin ny tany, satria, tsy misy zaza hitomany, 
fa momba Rasoabernanana, ka tsy niteraka Andriambahoaka afovoan' ny tany. 
Ary dia izao kosa no ataoko aminareo mivady, Andriambahoaka sy Rasoabemanana : 
Tsara izao, soa izao ny fiandriananareo ; kanefa tsy misy zaza hitomany ".


TOKO II.. DIA MAHAZENDANA

Nony nandre izany Rasoabemanana, dia nihobahoba (3) ka nitomany sy nikaikaika, 
nanao ranomaso havozona tamin' Andriambahoaka vadiny, sady nanao hoe : 
"Andriambahoaka ô, vory lahy ity ny fananana, ary be izato ny harena, fa voatavo (4) tokana no sisa tsy ananana, 
ka dia holovàn-dambo amin' alika ny tany amam-panjakana !"

«lzaho tsy hampirafy", hoy Andriambahoaka vadiny, fa mandehana hianao mamonjy an-dRanakombe, 
(5) makà ody zaza aminy. Dia lasa Rasoabemanana narahim-behivavy folo nitondra vato sy lehilahy zato
nitondra saboha (6) sy ampingaharatra (7) nankany amin-dRanakombe.
Vantany vao nahita azy Ranakombe. dia niteny hoe : E, lahy ! tsy tongotro mandia alanana, 
(8) tsy tanako mandray joria; (9) fa ny ho avy taona any hitako taon' ito, ary ny ho avy ampitso hitako anio. 
Ary fantatro fa na dia tsy miloa-bava aza hianao, ny hila zaza no alehanao sy mahory anao. 
Ka firy avy ny lehilahy sy vehivavy miaraka aminao ? "

Dia hoy izy: "Lehilahy zato mitondra saboha sy ampingaharatra zato, ary vehivavy
folo mitondra vato kiboribory folo".

«Ey, laby ! hoy Ranokombe, e lahy sarotra amin' ny tany izao, ka arivo lahy an-troka, 
(10) zato lahy anaty, ka folo taona hianao hivesatra azy ihany. Koa raha tianao izao, dia' homeko anao ; 
fa raha tsy tianao, dia tsy hanana hianao; ka dia modia, fa zaza loza izao,zaza antambo izao; soriba(11)

___________________________________________________________________________________________________
(6) Anaray ombiny daholo izany, hatramin' ny Igogoka ka hatramin" Imandralfakoandoha. 
(7) Vositra

Toko II
_________________________________________________________________________________________________
(3) Somebjseby. 
(4) Zaza. 
(5) Ranakombe dia vazimba mpanandro malaza indrindra tao atsinanan' Ambohímiangara avaratr Itasy, hono. 
(8) Lelona. 
(7) Basy 
(8) Laharan-tsikidy atao roa an-dalana. Labaran-tsikdy atao efatra an-dalana 
(10) Kibo.
(11) Fantara.