
SASINTENY

Ny ankabeazan'ireto "ANGANON'NY NTAOLO" sy ny fomban-drazana voalaza amin'ity boky ity dia nangonin-dRévérend 
L. Dahle sy ny misionera norveziana namany sasany, tokony ho telo-polo taona lasa izay, ka efa ela no nahalaniany. 
Ka dia nanonta indray izy tamin'ny Janvier 1908. Nalaky lany anefa ny natao tamin'izany, 
ka dia nohararaotiko ny fahalaniany handaharana sy hanitsiana ary hanampiana indray ny natonta fanindroany.

Ny anton'ny nanangonana azy dia ny mba hiarovana ny fomban-drazana tsy ho very tadidy ; 
ny hampandroso ny fahalalana teny malagasy; ary ny hampiseho ny toe-tşaina amam-panahy nananan' ny razana.
Mahagaga mihitsy ny angano sasany voalaza amin'ity boky ity, 
indrindra ange fa ny tantaran'lboniamasoboniamanoro, izay mampiseho
ta ny razana dia nino tsara fa misy fiainana any an-koatra; ny tantaran'Indesoka, 
zanak Itrimobe, izay mifandraidraika amin'ny an'i Samsona sy
Delila voalaza ao amin'ny Mpitsara, toko faha-16; ary ny tantaran faralahy Diso Fangataka, 
izay misy filazana ny Zanak' Andriamanitra nisolo heloka ny olombelona.

Raha dinihintsika tsara ny teny voasoratra ato, dia hiaiky volana isika fa nahay nandaha-teny mihitsy ny razana,
ary ampy hanaranam-po tokoa ny teny Malagasy. 
Firy amintsika ankehitriny moa no mahay mandaha-teny feno ohatrohatra toy izao :

"Aza mba manao tsimbadi-boanampotsy, lehiretsy, ka ho azon'ny olona tambatambazam-bola 
aman-karena hivadika ny marina. Fa ny daka no misy ambadika; 
ny loka no, misy ankoatra; ny akanjo no Soa misy paosy ; ny trano no tsara misy efitra ; 
ny tambin-toko no atao an' ankina ; ny salaka no atao sisik'ila ; 
ary ny hofak' ondry no soa mitsim-badika; fa ratsy koa raha ratsy, lehiretsy, raha mba mivadika ny marina." 
(anatra fanaon' ny Ntaolo N°1).

Mampalahelo anefa fa toa heverin ny tanora fanahy sasany bo tsy ampy intsony ny teny malagasy, 
hany ka mitady hampiditra baiko be ihany izy hampiseho ny fahaizany kanjo izany ataony izany no manambara fa
tsy mahay ny tenin-drazany izy.

Maro hianareo tanora fanahy no hamaky ny angano voasoratra ato ka ho finaritra sy ho dangidangin'ny hehy erỳ; ary ho hitanareo fa ny razana dia nidera ny fahamarinana sy ny hatsaram-panahy, ary nankahala ny
lainga sy ny halatra, ets. Maro koa anefa ny fomba tsy mahomby nahazatra azy, na tamin'ny teny naloaky ny vavany, na tamin'ny fiaimpiainany andavanandro. Tsarovy anefa fa tsy mbola mba nohazavain' ny Soratra
Masina izy tamin'ny nanaovany izany rehetra izany, koa dia tokony hofa-dintsika izay efa mahalala ny fampianarana nataon' i Jesosy Kristy ireny
fomba ratsy fanaony ireny.

Maro ny sakaizako, dia ny mpitandrina sy ny mpampianatra sasany eto Antananarivo, 
no efa nanampy ahy tamin' ny nanamboarana ity boky ity, indrindra fa F. Rasoamanana sy Rainiboanary. 
Tsy ho vita toy izao izy tsy akory raha tsy ny fanampiana azoko taminy. Sitraka sy telina ny nataony, 
ka te-hisaotra azy indrindra aho.

Hoy J. SIMS