an-tany; mivembena ny diany. Rasoabemanana no ataony menary anaty,
fa ny tsy manan-tsoa avy amin' izany no manenina>.

TOKO VII..- LALAO MAHERY SETRA

Rehefa tapitra ny resaka nifamalian 'izy mianaka, dia lasa nody ho any
amin' ny fonenany Ranakombe, fa Ibonia nijanona teo amin-drainy sy reniny
ihany. Nisy ankizivavin' Ibemampanjaka "efa-bavy niara-niteráka tamin-
dRasoabemanana, ka lahy avokoa ny zanany. Dia natolotr Ibemampanjaka
an' Ibonia ireo andevokeliny, ka niara-nitombo sy nilalao taminy. Ary nony
efa lehibe naharo tanàna izy dimy lahy, dia lasa niara-nilalao tamin' ny
ankizy tany an-tsaha, ka nifamely tain 'omby. Nizara nifanao an-daniny avy
izy ireo; nefa tsy nety Ibonia, fa nirehaka hoe : Aoka hianareo rehetra ho
an-daniny, ary izahay dimy miandevo kosa ho an-daniny>. Fa hoy ny nava-
lin' ny ankizy : <Ka hianareo vitsy ireo dia hahomby anay l>

Nefa tsy nety ihany Ibonia, fa namaly hoe : 'Ndeha hianareo misinda
er) an-kilany, fa aoka ihany izahay no ho an-daniny». Dia raikitra nifano-
raka izy, ka izay nasian' Ibonia, ny voa dia potraka, ary ny tsy voa fanina
nandraingiraingy. Koa resin' izy dimy miandevo ny ankizy rehetra.

Nony nody tany an-tanàna izy ired dia samy nilaza tamin-drainy
aman-dreniny hoe «Nifamely tain' omby mantsy izahay mhamin-dry
Ibonia, nefa izy dimy miandevo ihany no an-daniny, ary izahay ankizy
rehetra eto an-tanàna Ileolava kosa an-daniny, ka tsy nabaleo azy izahay
rehetra>. Dia hoy ny rainy aman-dreniny : «Edrey ! izany hakanosanareo
ankizy ! Ankizy tsy fanome hanina akory izany hianareo izany, raba izay
dimy lahy monja ka maharesy anareo rehetra>. Fa hoy ny avalin' ny
ankizy: Mba mankanesa ange hianareo any an-tsaha hahita ny afitsok
Ibonia>.

Nony ampitso dia lasa hizaha ny lalaon' ny ankizy tany an-tsaha ny
ray aman-dreny, ka nifamely loha-tany izy, nefa na dia maro aza ireo tsy
nahomby an' ibonia dimy miandevo; ka gaga foana ny olon-dehibe, raha
nabita ry fitorajofon' ny nasian "Ibonia.

Ary na inona na inona lalao natao, na nifampitora-bato, na nanao bala-
hazo, (l) na moraingy, (2) dia tsy nahaleo azy dimy miandevo izy rehetra.
Ary nony farany nanao vikina izy : nanezaka ny sasany ka nahafaka faharoa,
ny sasany faharoa sy mamakitratra, ary ny sasany fahatelo aza. Fa nony
mba nanezaka kosa Ibonia dia niboatra bonga iray mihitsy. Rehefa tafa-
verina tany an-tanàna ireo zatovolahy dia nilaza tamin' ny olona hoe :
«Nanao vikina mantsy izahay mbamin-dry Ibonia, dia tafahoatra irỳ bonga
irỳ izy, ka tsy hitanay izay nalehany». Nitolagaga foana teo ny olona nony
nandre izany ka nanao hoe : <Mahita loza isika raha very na maty any ny
zanak andriana>. Fa nony nahare izany kosa ny rainy aman-dreniny dia
tsy nanahy, satria fantany ny satany. Rehefa afaka hateloana dia inty tafa-
verina soa amAn-tsara izy.

TOKO VIII. -PORITRA KA NANAO HATSARANA

Rehefa lehibe Ibonia dia tsy nilalao tamin' ny ankizy tany an-tsaha
intsony. Ireo ahdevolahiny mpanaraka azy ireo dia novany anarana : llay

_________________________________________________________________________________________________
(1) Fanala fia. 
(2) Mifamely totohondry.