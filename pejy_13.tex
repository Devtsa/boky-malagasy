TOKO V. - SAROTRA OMENA ANARANA

Dia lasa Ihalampatana andevony nampitaitra an' Ibemampanjaka hoe :
«Teraka Rasoabemanana vadinao, tompoko, nefa maty eo ambony fiketra-
hana izy». Taitra tery Ibemampanjaka nony nandre izany, ka nanao hoe :
«Antsoy faingana ahy Itsiaronantso, ka asaovy maka arivo vavy hampandro
izao zaza izao>,

Rehefa tonga ireo vehivavy arivo avy tao an-tampon' Ileolava, dia
niditra tao an-trano mba handray ny zaza. Fa nony niakatra ny anankiray
handray azy, dia notsipahiny, ka folaka ny feny. Niakatra koa ny anankiray
dia notsipahiny, ka potsitra ny masony; ary ny sasany afa-nity; ny sasany
tapa-tanana. Dia rainy no avy hampandro azy, kanjo notsipahiny, ka folaka
terỳ ny teny. Dia tezıtra rainy ka niloa-bava hoe : *Endrey zaza loza,
zaza antambo ity, fa tan-kibo nanoto an-dreniny; latsaka an-tany, manoto
an-drainy I>

Dia niala teo amin' ny fiketraham-bolamena ny zaza ka nitsipika teo
anatin' ny motro (1) niroborobo, ary novonjena notondrahan-drano ny
motro, nefa tsy netyy maty, fa vao mainka nidedadeda. Nony noraisina ny
zaza, dia malama tsy voahazona; ary na dia fatratra aza ny firehitry ny
motro, dia tsy nampaninona azy akory. Dia hoy rainy tamin' Endriavelo,
rahavavin-dRasoabema nana : «Loza ity, ka 'ndeha faingana hianao mankany
amin-dRanakombe, ary mitondrà arivo. lahy milanja ampingaharatra sy
saboha, (2) ary folo vavy hiaraka aminao>. Dia lasa izy.

Fa sendra azon-tsindrimandry Ranakombe tamin' izay ka niteny hoe :
«Iha lahy izao manako mahamay ahy izao P Fa tsy toroana aho vao maha-
lala; tsy ilazana aho vao mahafantatra». Dia tonga tany aminy tamin' izay
Endriavelo, ka notsenainy teny hoe : «Fantatro ihany ny dianao, fa tsy
tongotro mandia alanana,° tsy tanako mandray joria,° fa teraka Rasoabe-
manana, ka namóno olona ativo sy nandratra olona zato ny zanany.. Ehe !
tsy izy izao, fa maniraha arivo lahy haka kitay, ka anampio lay teo mba
hiroborobo ny motro, fa mila anarana izao zaza izao l>

Dia lasa ilay iraka nanao araka ilay nolazainy, sady niainga koa
Ranakombe ka tonga tany Ileolava. Fa nony mby teo am-baravarana izy,
dia nitomany indray mandeha ny zaza ka tsy nivolana intsony. Rehefa
nahita izany Ranakombe, dia naniraka izy hoe : <Makà ombilahy efa-dahy
amín' Imangoronarivo, (3) ka vonoy amin 'ny lafin-tany efatra. Ary analao
efatra arivo lahy koa amin' Ileolava hanapoaka tafondro hampandroạna
izao zaza izao>. Dia novonoina ny omby sady nandefasana tafondro efatra
arivo vava ny lafin-tany efatra.

Tamin' izay dia nanonona ny anaran' ny zaza Ranakombe ka nanao
hoe :"«Akorakoray lahy, akorakoray lahy I fa izao no hataoko anaran' io
zazalahy io: Ipapangolahilavelatra: tsy ome-maka, ome-maka eo imason-
tompony; mipaoka eo imason-dray; maka ny eo am-pofoan-dreniny, na dia
ompa aza no ampanarahina azy vao mainka mahabe aina azy ihany.
Harihina toy ny tany nahalavoana; tsy mandrasa lavitra, fa mandrasa am-
bonin' ny loha. Lahý sarotra anie izao l» Nefa nandà ny zaza sady nihenjy
eran' ny trano, fa tsy tiany izany anarana izany.

_________________________________________________________________________________________________
(1) Afo. 
(2) Lefona • Jereo pejy faha-6. 
(3) Anaran' omby.