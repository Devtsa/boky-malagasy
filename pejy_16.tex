velona andavan-taona. Ny fanjakany tsy ananan' olona; ny fiandrianany tsy
iraisany amin' ny hafa, fa izy irery no antonona azy hatrany an-dafin ny
riaka manetriketika. Manatihitiby 'ny an' olona, fa tsy mba tihitihin' olona.
zafin' Andrianairery, anak' Itokamanantsoa. Tsy hitan' olona ho ama-
miady, izy no mampandry tany, fa lahy sarotra. Fa ny biby no biby, ny voay
no voay; fa izy no voay be miandry am-pitàna : setrain-daka-mandrendriká,
robohi-manehi-baniana, trano be tazana : tsy ahazoana maro an-entana, fa
raha ahazoana mandratra. Koa raha mitazantazana azy aza ny eny
an-dafin' ny riaka eny, dia manampy azy koa. Ombilahy sakoko tandroka
Ibonia ka sarotra iadiana: an-tendron-tandrony voatevika; diniridiriny, sola
ny loha; an-tsofiny voakapoka, an-drambony voafioka, ambany kitrony
voahitsany. Ny fofon' ainy manjary rambondanitra; ny tanany havanana
mandritra alin-dahy; ny tanany havia mandritra arivo lahy; ny ombilahin-
tongony mirehitra amin' ny tany. Mizaozao ny akata; taman-deboka (23) ny
tany ary may ny tanàna. Manitra omby ny vohiny; ka be varanga mody,
(24) sady tsy lany toaka. Toaka iray siny lahy ny an Tbonia ka mahamamo
arivo lahy.

TOKO VL - TOETRA MANAKA LAHY
Raha vantany vao tapaka ny tenin-dRanakombe, dia nihetsika Ibonia
tao anaty afo mivaivay, ka nipololotra ny varatra nisarika azy. Ary tonga
nifaoka azy koa ny rambondanitra; nikotrokotroka ny vato; nivadibadika ny
tany; ritra avokoa ny rano tany Mananivo; ka dia nanao ranonoram-baratra
ny andro; vaky nialoha ny entana tany amin-dRaivato, ilay naka an' lampe-
lasoamananoro; ary intelo tafatsipika izy vao tafarina.

Raha nahita izany Andriambahoabesofina (rain-dRaivato) dia nihiaka
ka nanao hoe : «lzao no kabary ataoko anminareo terak' i Mananivo : Ny
mpamboly dia mambole : ataovy ny voly rehetra ,na ny an-drano na ny
an-tanety, fa tonga ity ny maoraran (1)-taona>.

"Ehe !" hoy Raivato; «Ibaba lahy no atao ho mpahafantatra kanjo tsy
mahaləla; atao ho mpitadidy kanjo tisy mahatadidy. Tsy tongotro mandia
alanana, tsy tanako mandraý joria; fa izaho no atáo ho mpahalala ka ma-
halala, atao ho mpitadidiy ka mahatadidy. Tsy mba maoram-pahavelomana
izao; fa ny ho avy taona any hitako taon' ito, ary ny ho avy ampitso hitako
anio, Fa izao kosa no ambarako anareo, ry terak' i Mananivo : Tsy kotrok
andro mahavelona izao, tsy taom-pambolena, fa maoram-paty; koa ny ma
nam-bondraka (2) mamonoa; ny mana-mahia manakaloza; ta fandroan-dahy
sady ozatina no masin' ody izao. Vain' afo no androan' izao zaza izao, fa tsy
amonoana omby ,tsy amonoana akoho; tsy mba mandro ny rano androan' ny
olona.

«Koa somodasoda (3) ny oroko, zebizeby (4) Betsiboka, mandra-pita
tsy lakanina. Miverivery (5) lambo roa tany, fonjafonja ny vositra, marolily
(6) ny tamanana, (7), fola-nofo ny ombikely, tomboazily (8) ny ombilahy.
kolodoy (9) ny rain-kazo, mihaon-tendro ny tsipolitra, mizaozao ny akata.
Fa andriandahy sady ozatrozatina no masina ody aho. Nefa matin-dahy
iraika ity tany ity, ka Andriambahoabesofina no namono azy. Ny olona
mitovy aina amiko tsy mahafaty ity tany ity, saingy ny raha hanaovako azy

_________________________________________________________________________________________________
(23) Mihorohoro, mihovitrovitra. 
(24) Motika ka tsara.
Toko VI _ (1) Fotoana?
(2) Matavy
(3) Fenofeno 
(4) Tondraka
(5) Mivezivezy
(6) Mandim-bolo noho ny taviny. 
(7) Reniomby. 
(8) Botrefona, dongadonga. 
(9) Betsaka sady matevina?