ka ho biby tsy misy setriny, ary ho mpanjaka tokana, dia milatsaha
eo anaty afo!" Dia nanidina nankao anaty afo ny sompanga tamin izay
ka masaka. Ary tsy nesorin'olona izy, fa nitsipika niala ho azy teo am-patana
na dia nitsatoka teo an-tampon-dohan-dRasoabemanana, ka nitsarapaka
tany an-kibony, dia niforitra ka tonga zazakely.

Rehefa niforitra izy, dia novonoina ny ombilahy folo sy ny ondrilahy
zato ka lanin-dRasoabemanana samy irery avokoa; dia tsy nihinan-kanina
intsony izy mandra-pahaterany, fa ny rivotra ihany no notohofany.

Nony nandray telo taona an-kibo ny zaza, dia nifofo vady hoe:
"Vadiko Iampelasoamananoro; (5) ka dia ambeno tsara fandrao tongan-dahy sarotra".
Dia hoy reniny: "Akory izato hianao no mbola tsaika (6) ka mifofo vady sahady?"
Fa hoy ny navaliny: "Hitako fa tsaika aho, Adarà, fa fofoiko fahatsaika ho vadiko fahazaza;
koa tsy itiavako lahy, tsy anaranako azy, tsy hasoavako tany aman'olona.
Fa raha maty Iampelasoamananoro, dia tsy hairitro (7) aman-tany; ary raha velona,
tsy aritro ho an-dahy. Endre! fa misy mila loza mitady haka ny vadiko".

Dia namaly azy reniny hoe : "Meda! Ikabikabilahy angaha no sahy
hanao izany, fa izy no milom-pa (8) tsy many, milom-pa madio, tsy afatotra
an-tety, (9) tompoi-maro manana, fary be vololona, miontsy be fahy.
Ny tain-kohony ambony tsy lanin'amboa zato, ny tain-kohony ambany tsy lanin'amboa folo arivo,
ny feny an-ila fanasan-tsarika, ny feny an-ila vatoasana,
ny lohany riandriana fanasam-by, ny tanany tantanana, ny rantsany tandra,
ny laferany fatana, ny kibon-dranjony tafoforana, ny sandrin-tanany tsofa-lahy fanofam-by, 
ny volon-keliny ibangoan-trandraka, ny volon'orony ana-todizan'ny voro-madinika,
ny volo-masony ibongoan-tsokina. Na angamba Ifosalahibehatoka no sahy hanao izany, 
fa izy no mandidy ny tsy hohanina sy manokana ny tsy handrahoina hoe: "Atokony, atokony; loary, loary;
'zay mifira tsy amim-pamaky, ary mandidy tsy amin'ny miso. (10). Izany
no lahy sarotra amin'ny tany". Fa hoy ny famalin' ny zaza : "0! Adarà,
zaza izany ka tsy mahasakana ny loha-lambo hotorahako".

Dia hoy indray reniny "Angaha Impandrafitrandriamanibola sy
Andriambavitoalahy no hanao izany, fa izany no taranaky ny mpizara sy
zafin'ny mpanome, taranaky ny mpamono sy anatin'ny mpamelona, tarana-bolamena sy zafim-bolafotsy, 
ary taranaky ny mpihary sy zafin'Ibezezika, Koa ny tany diavin'izany dia tsy diavin-dahy mavozo,
(11) tsy diavin-dahy matahotra. Izany no lahy sarotra amin'ny tany. Fa tsy ny biby no biby, 
ary tsy ny voay no voay, fa izany no voay be am-pitàna; ka raha
setrain-dakana, mandrendrika; raha robohina, manehi-baniana. (12) Koa
ny tsy voan' izany no asa; fa raha azon'izany dia mipepipepy (13) miankanga asaly, 
(14) ka tsy hita izay atiny sy vohony". "Tsia, Adarà, hoy ny zaza, 
"fa tsy mamerin-dohomby horoahiko iny, fa mbola zaza".

Fa hoy indray Rasoabemanana: "Angaha Rainingezalahy sy Imbolahongeza no ho sahy izany,
fa ireny no biby, ireny no mamba, taranak'Impanarivo, zanak'Ibefaofao; arivo lahy mihina (15) ireny,
zato lahy mihaga. (16) Koa raha mitoraka lambo ireny, dia tsy mitoraka akaiky, fa
mihoatra saha fito vao mitoraka. Kanefa tsy sinda, (17) fa montsana ny

_________________________________________________________________________________________________
(5) Ny renin' Iampelasoamananoro dia Ravaomanana, havan' Ibemampanjaka, rain' Ibonia.
(6) Zaza tsy ary fady. 
(7) Halevina. 
(8) Mandro. 
(9) Ambony. 
(10) Antsy. 
(11) Osaosa,farofy. 
(12) Manaikí-baniana. 
(13) Mankatsy mankaroa 
(14) Atsatsika. 
(15) Mijihina, ingo,
(16) Mihaingo, mahery. 
(17) Mivena.