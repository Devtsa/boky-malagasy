valo arivo; lavo tsy aman-tady ny vositra; ny ombiaby lavo ho azy dia
ateliny tsy mirasa. Izy no mifoha amin' antsiva, mandry amin-kazolaby:
izy no mamoha lalana ny andro ho andro, mahalala ny alina ho alinas.
"Tsy tiako izany, ry baba», hoy izy, «fa sikin-dahy mavozO : mandeha tsy
mahataka-davitra, mandeha tsy mahatokan-tany, mpitsioka (17) isaky ny
vody hazo",

"Hataoko hoe Isikindahinimpandrafitrandriamanibola àry no anaran'
ialahy, milalao misikin- dahy anila-tevana. Ny haranany itarafan' ny kibony:
ny rirany toa ranomanga; ny endriny roa toa riam-bato; ny vaniany toa tady
vola; ny voavitsiny toa andry be voavankona. Izy no manana arivo lahy
mihina, zato lahy mihaga; isao-be niarahana ny sikin-dahiny, arivo vatritra,
(18) zato fandraisana; (19) tsy disan' ampela mipaopaoka, fa ampela
matotra ihany no mandray raha manamboatra ny sikin-dahin' Ibonia>, <Tsy
tiako izany, ry baba>, hoy izy.

«Hataoko hoe Isikindahiningarabelahy àry no anaran' ialahy : ny lelany
fihai-drivotra, ny tenatenany hoatra ny ranomandry, ny salakany isampazan-
jato lahy. Izy no biby sarotra amin' ny tany toa ahy>., <Tsy tiako izany,
ry baba, hoy izy, <ta tsy mamono mahafaty, fa nofirain' olona, tsy maha-
faka mora, tsy mandratra ny sarotra. Mitoraka miamboho, mamono am-pı-
taka; mitoraka tsy mahara-dahy izany, ka tsy tiako>.

"Hataoko hoe Isikindahinandriandambozoma àry no anaran' ialahy;
fa somolasola (20) ny loha salakany; misava hivoka, mibohibohy ny an-
daniny roa; manipaka ny akata, mipika ny rambon-tsalakany, manoto ny
híratra, (21) mandia ny kitrokeliny; mitarehim-boavato (22) ny rambon-
tsalakany. Izy no dongidongy, izy no kazana, (23) izy no petra, (24) izy no
mandidy, izy no saborobotaka, izy no andrianevoka, (25) izy no takalitra;
(26) ny tsy voa no asa, a ny voany dia saborobotaka, fery be tsy taham-
bady masay. (27) Fa ny tsy voany hoatra ny lalovany; napiny izay nokina-
siky ny hety Vazahany; mitaifay be ihany ny tsy voa. Silkin-dahy sarotra
anie izany, anaka è !> <Tsy tiako izany izäny, ry baba», hoy izy, cfa sikin
dahy diso tany izany : tsy mahatarika mpanjaka; tsy mahajery vahoaka; tsy
mahadimby tanin-dray; tsy mahatan-dovan-tena; tsy hita fototra izay fone
nany, fa ny alin-kandriana no ihazany, ary ny andro kosa no atoriany; tsy
manana anton-koraka, fa halatra no antenainy; saonjo mangidy no vatsiny>.

«Hatacko hoe Isikindahinandriambavitoalahy àry no anaran' ialahy:
mkorefa miafara, volafotsy no sikin-dahiny, volamena no fehy vaniany,
manao sikin-dahy vasia (28) mandraka alina izy; valo volana an-kibo. Izany
no masoandro ataony anaty : miasa, misavily tanana; tsy vaky andro arara;
(29) tsy tsofotra andro tsy miberina; aivo lahy no manantsa (30) azy, alin-
dahy no manomp0, tsy fotoanin-dahy ratsy, 'tsy sangin-dahy mavozo, ny
tany diaviny mibaibay foana ka tsy vanon-ko raha. Ny masony mananon-
tanona, ny handriny volan-tsinana, ny nifiny anakandry voavankona, ny
voavitsiny fantara roa latsaka an-tany; ny tongony antsoro (31) avy any

_________________________________________________________________________________________________
(17) Maka aina. (18) Taolana: (19) Tanana, (20) Simbasimba. (21) Maso. (22) Anaram-
boankazo, (23) Mazana, (24) Didy. (25) Mievoka. (26) Kalita. (27) Ny vady faharoa amin' ny
mampirafy. Tsy taham-bady masay = Na día ny. vady masay aza no atao taha. (28) Xara-
zan-dalao. (29) Fatratra, malaza. (30) Miantsa, míboby andriana." (31) Angady.