TOKO III, - SOMPANGA NANJARY ZAZA

Nony tonga tany an-tanànany Rasoabemanana, dia lasa ny arivo lahy
naka kitay hitonoana ny sompanga hohanin-dRasoabemanana. Nony tonga
ny kitay dia natao indray mandrehitra, ka natono teo ny sompanga; kanefa
nivaivay foana teo anaty afo teo izy fa tsy nety masaka, na dia efa loza
aza ny fidedadedan'ny afo. Dia naniraka naka an-dRanokombe izy. Kanjo
raha tsy mbola tafapaka tany akory ny iraka, dia azon-tsindrimandry izy
ka niloa-bava hoe: "Ey! injao, ry lampelafetsy, (1) mba tazano ao atsinanana ao. 
Tsy tongotro mandia alanana, tsy tanako mandray joria ; nefa ny
ho avy taona any hitako taon'ito, ny ho avy rahampitso hitako anio"

Dia lasa nitazana izy. "Indreto", hoy izy, "misy fito lahy avy
atsinanana miolo-may aman-defona, mihindrahindra amin'ampingaharatra". 
"E!"" hoy Ranakombe, "irak lampelamanandahiloza ireny", miantso an-dRasoabemanana.

Noho izany, rehefa tafapaka tany aminy ny iraka, dia hoy izy: "Tsy
toroa-mahalala aho, tsy ilaza-mahafantatra.

Firy lahy no nitsoka antsiva?
Firy lahy no nanapoaka ampingaharatra? 
Firy lahy no nively hazolahy?
Firy lahy no nampiady ombilahy manodidina ny trano ? 
Firy lahy no naka kitay ?
Firy no ombilahy novonoina.? "
Dia indray namaly ny iraka hoe:
« Tsy teo izany, abakobe ".

Dia hoy indray Ranakombe taminy Ndeha aloha hianareo mody,
ka mamoria ombilahy arivo hampiadina sy ampingaharatra arivo hapoaka
sy antsiva arivo hotsofina sy mpively hazolahy fito-polo, ary ombilahy
intelo fito sy ondrilahy zato hovonoina. Fa izaho mbola hiandry ny andro
Zoma, izay vitan' Alahamady tsy mahafoy Mpanjaka, ka mampanito (2)
an-dRaondriana, ary mahato volana an-dohavohitra tsy mahafaty olo-mainty. 
Mahery vintana izao zaza izao, koa tsy hotampohiko andro, tsy
hotebahiko taona, fandrao manolaka ahy ; koa 'ndeha aloha hianareo
miverina any amin-dRasoabemanana.

Rehefa tonga tany amin-dRasoabemanana ireo iraka, dia novoriny
ireo zavatra ireo mandra-pihavin-dRanakombe. Nony tonga ilay andro
Zoma, dia niainga ho any izy, ka nony tonga teo am-bavahady dia nanao
hoe: "Apoahy manodidina ny tanàna ny tafondro, ary Itokamahasoitra
sy Itokamarovonoina (3) ary ny ampingahara-bolamena dia atsangano
ao amin' ny tehezam-bohitra atsinanana". Dia napoaka manodidina ny
tanàna ny tafondro, sady nampiadina ny ombilahy arivo ka nakoraina
sy nohobena. Ary novonoina koa ny ondrilahy zato sy ny ombilahy intelo
fito. Ny anton'ny anaovana an'Itokamahasoitra sy Itokamarovonoina eo
amin'ny ila-tanàna atsinanana, satria ao ny rafin'ny zaza.

Izay vao niditra tany amin-dRasoabemanana Ranakombe. Ary nony
vao ningadona teo an-tokonana izy, dia nionbotra niala teo anaty fatana
ny sompanga ka tafapaka terỳ amin ny vovonana; ary dia nipariaka eran'ny trano ny afo.
Dia nibaiko Ranakombe ka nanao hoe: "Hitako fa zaza loza; hitako fa zaza antambo; 
hitako fa lahy sarotra (4) amin'ny tany Ifatsietra; nefa na izany aza, 
nomeko an-dRasoabemanana teo an-tampon'Ileolava ihany. Kanefa raha tsy ho vanona, 
dia mandehana mianatsimo na mianavaratra, fa aza milatsaka eo anaty afo. 
Fa raha ho vanona dia ho vanona kosa
___________________________________________________________________________________
(1) Andevon-dRanakombe io. 
(2) Mampanjaka. 
(3) Anaran-tsampy ireo. 
(4) Loza, mahery.