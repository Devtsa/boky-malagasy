izao, varatra izao. An-tany manola-dray, (12) an-kibo manola-dreny".
Nony nahare izany izy, dia hoy ny navaliny: "Eny, lahy! fa ny tsy mananko loza no ratsy, 
ka aleo ihany mananan-kandova, na dia antambo aza ".

Dia namaly Ranakombe hoe: "Eny ary, fa raha tianao ihany izao,
mandehana hianao mankany amin' Ivatolahiarivozoro. Samy ho tonga ao
avokoa ny biby sy ny zava-mahafaty rehetra, ary hilatsaka ao koa ny varatra. 
Kanefa raha vao hiainga hianao, dia asaivo mitondra vaton-tafon-dro roa avy ny vehivavy manaraka anao, 
fa hianao koa mitondrà telo; ary sompanga no ody zaza halainao ao.
Arivo lahy mifaifay (13) no ho tojo (14) anao ao; 
nefa ho vaky mandositra avokoa ireny, fa sompanga anankiray lahy no hanao omby maola (15) ao, 
ka hanodidina an' Ivatolahiarivozoro.

Dia niainga nankany izy ireo; ary nony tonga teo akaikin'ilay vato,
dia nofinaoky ny rambondanitra sy ny rivotra sy ny soriba sy ny varatra
sy ny havandra ary ny zava-mahafaty rehetra, ka dia im-pito lavo niantombina izy folo vavy nanaraka azy,
fa Rasoabemanana irery ihany no tsy mba lavo. 
Ary nony nanatona ilay vato izy, dia niseho indray ny zava-mahafaty rehetra, 
ka nony tonga teo avarany izy, dia nisy sompanga nipețraka teo an-tampon' ny vato, 
ka niletsy io sady tonga somafatra (16) tamin' ny tany, 
dia lavo indray izy folo vavy sady folaka avokoa ny tanany, fa Rasoabermanana mbola tsy lavo ihany. 
Dia notontana ny vaton-tafondro ny sompanga mba halain-dRasoabemanana, nefa tsy nety maty, fa mainka
velona nangaihay. (17) Dia nisikin-dahy izy handray azy, kanjo nony azony dia samy nentin' Ivatolahiarivozoro 
nanidina izy, ka kely sisa tsy nitehika tamin' ny lanitra; kanefa tsy nety potraka, 
fa tafapetraka tao an-tampon'ny vato ihany. Dia nidina tamin'ny tany indray ny vato, ka nitsatoka
teo amin'ilay nitoerany ihany; ary nilentika tao anaty vato Rasoabemanana haka ny sompanga hataony ody zaza. 
Dia samy niteny avokoa ny vahatry ny hazo (18) tambonin'ny tany nanao hoe: "Ezaho no ody zaza, izaho no ody zaza."

Nony tonga teo amin' ireo ody zaza ireo Rasoabemanana, dia Itsimala,
zoafindra sy Itsimainambolena sy larivoantandrokosy ary Izatoankibo, dia
niditra tany anaty hazo izy haka an'Itokambololona. Nefa nony tonga
tany an-tampon'Itokambololona izy, dia ny sompanga no nandray azy.
Dia nandeha nitety tamin' ny faroratra izy ka tsy nitehiką ny lanitra, tsy
nikasika ny ravin-kazo, ary tsy nandia tany.

Ny ody zaza nalainy tany an-ala dia nanindry an' Andriambahoaka
vadiny tany an-tanàna hoe : "Miraria hianareo, fa miady Rasoabemanana.
Ary izao no ataovy firarinareo :

"Ento, indaosy atỳ ny lehilahinay,
Tsy atao hivoka ny amalon-kely loha,
Tsy atao am-pefy rano>, (19).

Dia niverina Rasoabemanana ka nitsatoka tany amin'ny vohiny atao
hoe Ileolava; koa dia nihotrakotraka ny vato, nizaozao ny akata, ary nanjary
nahaleo nahalasa ny tanàna, ka izany no nanaovana azy hoe Ileolava.
_____________________________________________________________________________________
(12) Manoto. 
(13) Manan-kery. 
(14) Mifanena. 
(15) Maditra. 
(16) Mitovy tantana. 
(17) Niharihary. 
(18) Fakan'ny hazo 
(19) Tsy voasakan-drano.