Manan-tombo anie izao, Adarà; raha fotoanina izao tsy tratra, mamotoa-
mahatratra; mitaky mahazo, itakian-tsy ahazoana; fa raha vonoiko an-davan-
taona, dia maty an-davan-taona, velomiko an-davan-taona, dia velona
andavan-taona. Mora atao velona, fa sarotra atao maty. Vatako izaho irery
ka foloalindahy no mihina, arivo lahy no manantsa, (15) zato lahy no
mihaga; tsy tafihin-dakana arivo aho, tsy alain-dakan-jato». Dia hoy reniny:
<Vatanao samy irery ve, anaka, ka dia tsy ho zakan-dakana arivo, tsy ho
zakan-dakan-jato? Ha ! an-troka (16) io hanola-dreny, an-tany hanola-
dray !»

Fa hoy indray ny zaza : «Izaho tsy manatihitihy (17) ny tia ahy, tsy
ratsy volana amin 'ny malaina ahy : fa ny manatihitihy ahy no tihitihiko,
(18) ny volana amiko no mba iratsiaka volana; raha soa volana amiko dia
mba havaniko ,na dia ny any an-dafin' ny riaka sy ny any ambonin' ny
lanitra aza. Fa izaho tsy namboarin' olon-tongotra, na namboarin' olon-
tanana, fa ny tany ahiko, ary ny lanitra ahin' Andriamanitra; fa Izy Andria-
manitra any ambony, ary izaho andriamanitra atỳ an-tany. Koa raha soasoa
volana amiko mba isoasóaviko volana; ary raha ratsiratsy volana amiko mba
iratsiratsiako volana; fa izaho no toka-mahasesi-tany».

Dia naniraka an-dreniny izy hitsidika ny andro, ka hoy reniny rehefa
avy nijery : «Mitondra vovonana ny andro». «Saka mandroba izao>, Adarà,
hoy ny zaza,: <mandroba ny lanitra izao, mandroba ny atsinanana izao,
mandroba ny andrefana izao, mandroba ny atsimo įzao; ka dia anampiko
ahy avy izao. Apoahy ny basy efatra arivo vava amin' ny lafin-tany efatra,
nefa anapoahy arivo vava aloha mankany an-danitra>. Dia nampanaovin-
dreniny izany, ka nony efa vita, dia hoy ny zaza : «Lehilahy izany, Adarà,
fa niady hafoloana ka apoahy amin' izao ny amin' ny lafn-tany efatra>.
Dia napoaka, ka nanontany izy na efa napoaka na tsia. Dia hoy ny navalin-
dreniny : Tsy nisy nety nipoaka, afa-tsy iray vava tatsy atsinanana>. <Veli-
rano amim-bahoaka izao, Adarà; azontsika ny tany izao, fa tsy nisy basy
nipoaka; ary tolona indray andro mantsana (19) izáo. Ho teraká amín 'izao
aho, Adarà, fa inty ny fiketrehana handrianao, ary iroa ny fketreham-bola-
mena kosa hitaingenako : fahafito an-tsakany, fahefatra ambin' ny folo
an-davany.

Nony efa ho teraka tokoa ny zaza, dia niteny izy hoe : «Alao ny motro,
(20) ka ataovy mirehitra be am-patana; ary anefeo antsy fiharatra aho, ka
atelemo amin' ny akondro, fa izaho tsy hiboaka miorika na hiboaka miva-
lana, fa hiboaka eto an-trokanao, Adarà>.

Rehefa nandre izany Ibemampanjaka vadiny, dia nalahelo ka loa-
bolana hoe : <Tsy izao va no tsy tiako !» Fa hoy ny navalin' ny vavy : «Ny
tsy manan-doza no ratsy, fa ny zaza no mpandova ray>. Koa dia nanefy
antsy fiháratra Ibemampanjaka ka nataony táo anatin' ny akondro ary
natelin' ny vadiny. Dia lasa izy nivoaka. Rehefa izany dia noraisin' ny zaza
ny antsy, ka raha mbola teo ambonin' ny fiketraham-bolamena reniny, dia
notatahany ny kibony hivoahany, ka nitsipika nitaingina teo ambonin' ny
fiketraham-bolamena izy. Dia maty teo reniny, ka nataony niantsinanan-
doha. Tamin' ny andro nahaterahany dia fola-pe avokoa ny zava-manana
aina rehetra na inona na inona, nitresatresaka ny vato, nivadibadika ny
tany, ary nikotrokotroka ny lanitra. Izany, hono, no nisehoan' ny horohor-
tany.

_________________________________________________________________________________________________
(15) Miantsa. 
(16) Kibo. 
(17) Manao olona «Ilay ity. 
(18) Sahala amín" ny hoe: «Mandritika ahy mba ritihiko; raha maneva ahy mba hevaiko». 
(19) Maninjitra. 
(20) Afo.