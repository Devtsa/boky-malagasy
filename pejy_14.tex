<Meda lahy izao zaza jzao l> hoy Ranakombe, «hataoko hoe Andriam-
boromahery no anarany : mifaofao (4) ka tsy maka afa-tsy eo imason' ny
arivo lahy, tsy maka eo imason' ny olom-bitsy; koa ny tsy voa tsy vanon-ko
raha, ary ny iafarany mibaibay; (5) ny an-tanany tsy latsaka; izany no toka-
mahamamo tany, Tiako anie izany, ry baba>, hoy ny zaza; <nefa mpa-
ngalatra sakan' akoho ireny, ka izany indray no tsy itiavako azy>.

«Akorakoray lahy, akorakoray lahy !» hoy Ranakombe, efa
Andriandambozoma no anarany, izay mikotrokotro-tsy aman-kitra, (6)
mamelona afo tsy am-poserana, (7) mikapakapa tsy am-pamaky, mihady
tsy am-pangady, mivalan-tsaha tsy misava hivoka, (8) mitety bonga tsy
amin-kapa, kibotry (9) valon-jato, vahatr ampanga valo arivo: ny tsy voa
no asa, fa ny voa miroraka tsy vanon-ko raha>. «Ehe I tsy tiako izany, ry
baba !» hoy ny zaža.

«Meda lahy izao zaza izao !b hoy Ranakombe; <hataoko Andraifosa-
lahibehatoka no anarany: mitratra-palahy, mivaniandahy sarotra, mitongotr
Impanenjidahy, mifeo-mpamono samy irery; maka ny ambanin' ny famaky
(10) toy izay ny ambanin' ny fe ,tsy homana ny ventiny fa mitroka ny rany.
Izany no lahy sarotra amin 'ny tany. Koa raha vonoin' izany andavan'
andro, dia maty andavan' andro; raha vonoin' izany andavan' alina, dia
maty andavan' alina». «Tsy tiako izany, ry baba>, hoy ny zaza.

«Akorakoray lahy, akorakoray lahyl» hoy Ranakombe, efa ny anaran'
izao zanakao izao hataoko hoe Andriantolohoboboka : misidin-tsy avo,
nandeha tsy be, mitsororo-pihinana, (11) manta fanao barandro. (í2) Ny
1ilo volany zato, ny fitrebika (13) arivo; mifoha amin' antsiva izy; mitelina
ombilahy (14) iray. Izany no lahy sarotra amin' ny tany. «Tsy tiako izany,
ry baba, hoy ny zaza, '«fa midian-olom-botry>.

<Edrey, meda lahy izao zaza izao l» hoy Ranakombe. «Ny anaran' izao
hataoko hoe Andriamitomoamibotretraka, (15) arivo lahy mamohotra, foloa-
lindahy mihaga, zato lahy mihina; akorakora ihany no entiny, ka mahavaky
tany ny hobiny». «Ehe I tsy tiako izany>, hoy ny zaza, «fa lavo antonon-
tonony, ka ny mavozo mandia loha, ny osa mandalo vady; tsy tiako akory
izany, ry babas.

Dia hoy Ranakombe : «Akorakoray lahy, akorakoray lahy I ny anaran'
izao zanakao izao hataoko hoe Andriamborondolosarotralinafamorandro; ka
ny andro ataony alina, ary ny alina ataony andro; mitelina omby (16) samy
irery, nefa tsy mitondra antsy handidiana, tsy mila famaky hikapana>. «Tsy
tiako koa izany, ry baba>, hoy ny zaza.

«Meda lahy izao zaza izao hoy Ranakombe. «Ny anaran' izao
zanako io hataoko hoe Andriankabikabilahy : milom-pa tsy many, milom-pa
madio ,tsy afatotra an-tety, (17) tompoi-maro manana, miontsy be fahy,
fary be vololona. Ny tain-kohony ambony tsy. lanin' amboa zato; ny tain-
kohony ambany tsy lanin' amboa folo arivo; ny volon-keliny ibongoan
trandraka, ny volo-masony ibongoan-tsokina, ny volon' orony anatodizan'
ny voro-madinika, ny volon-dohany anatodizan' ný goaika, ny volon-katony
iterahan' ny akanga; ny feny an-ila fanasan-damba, ny feny an-ila fanasana
antsy, ny lohaliny riandriana, ny totohondriny tantanana, ny rantsan-tanany
tandra, ny ranjony tafoforana, lzany no mirehaka hoe : «Atokony, atokony;

_________________________________________________________________________________________________
(4) Mifaoka. 
(5) Mahamay, idiran-doza, 
(6) Tsy haikaina. 
(7) Hazo ampifandilorina bame-lona afo, 
(8) Tany tsy vaky lay. 
(9) Tany atao nongonongona. 
(10) Sahisahy manao zava tsarotratao. 
(11) Miadam-pihinana. 
(12) Toaka. 
(13) Fanjaitra. 
(14) Valala. 
(15) Sabona.
(16) Voalavo. 
(17) Ambony.